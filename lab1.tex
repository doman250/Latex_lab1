\documentclass[11pt,a4paper]{article}
\usepackage{polski}
\usepackage[utf8]{inputenc} 
\usepackage[dvipsnames]{xcolor}
\title{Sprawozdanie z IoT-Laboratorium nr1}
\author{Damian Domański}
\date{17.10.2018r.}

\begin{document}
\maketitle 
\section{Wstęp}\label{sec:begin}
LaTeX-jest to oprogramowanie służące do zautomatyzowanego składu tekstu,a także związany z nim język znaczników,służący do formatowania dokumentów tekstowych i tekstowo-graficznych(np.broszur,artykułów,książek,plakatów,prezentacji,a nawet stron HTML).
\section{Formatowanie czcionek}\label{sec:form}
Według rodzaju i grubości czcionek:\newline\newline
\textrm{pismo proste}\newline
\textsl{pismo proste pochylone}\newline
\textit{kursywa}\newline
\textbf{pismo pogrubione}\newline
\textsc{pismo z dużych liter}\newline
\texttt{pismo imitujące pismo maszynowe}\newline\newline
Według wielkości czcionek:\newline\newline
{\tiny Jestem najmniejszy}\newline
{\scriptsize Jestem bardzo mały}\newline
{\footnotesize Jestem nieco większy}\newline
{\small Jestem mały}\newline
{\normalsize Jestem o rozmiarze normalnym}\newline
{\large Jestem duży}\newline
{\Large Jestem bardzo duży}\newline
{\LARGE Jestem większy}\newline
{\huge Jestem jeszcze większy}\newline
{\Huge Jestem największy}\newline
\section{Kolorowanie czcionek}\label{sec:kolor}
W tym punkcie zostanie przedstawiona sytuacja w której kolor czcionki zmieni się z czarnego na różny kolor.Oto przykłady:\newline
{\color{red}czerwony,}
{\color{green}zielony,}
{\color{yellow}żółty,}
{\color{blue}niebieski.}\newline
Wczoraj jeździłem {\color{blue}niebieskim motocyklem},który miał światła stopu {\color{red}koloru czerwonego}i światła mijania {\color{yellow}koloru żółtego}.\newline Po drodze minąłem {\color{violet}fioletowy dom} o {\color{green}zielonym podwórku.}\newline Oto jest {\color{gray}moja historia napisana szarym kolorem.}\newline\newline
\section{Tabele}\label{sec:table}
W tej części przedstawię różne rodzaje tabel m.in.jednowymiarową,dwuwymiarową itd.\newline Oto przykłady tych tabel:\newline
-tabela jednowymiarowa(w stronę wierszy):\newline

\begin{tabular}{|l|l|l|l|l|}
\hline
1 & 2 & 3 & 4 & 5 \\ \hline
\end{tabular}

-tabela jednowymiarowa(w stronę kolumny):\newline

\begin{tabular}{|l|}
\hline
1 \\ \hline
2 \\ \hline
3 \\ \hline
4 \\ \hline
5 \\ \hline
\end{tabular}

-tabela dwuwymiarowa(przykład tabliczki mnożenia):\newline

\begin{tabular}{|l|l|l|l|l|l|l|l|l|l|}
\hline
1  & 2  & 3  & 4  & 5  & 6  & 7  & 8  & 9  & 10  \\ \hline
2  & 4  & 6  & 8  & 10 & 12 & 14 & 16 & 18 & 20  \\ \hline
3  & 6  & 9  & 12 & 15 & 18 & 21 & 24 & 27 & 30  \\ \hline
4  & 8  & 12 & 16 & 20 & 24 & 28 & 32 & 36 & 40  \\ \hline
5  & 10 & 15 & 20 & 25 & 30 & 35 & 40 & 45 & 50  \\ \hline
6  & 12 & 18 & 24 & 30 & 36 & 42 & 48 & 54 & 60  \\ \hline
7  & 14 & 21 & 28 & 35 & 42 & 49 & 56 & 63 & 70  \\ \hline
8  & 16 & 24 & 32 & 40 & 48 & 56 & 64 & 72 & 80  \\ \hline
9  & 18 & 27 & 36 & 45 & 54 & 63 & 72 & 81 & 90  \\ \hline
10 & 20 & 30 & 40 & 50 & 60 & 70 & 80 & 90 & 100 \\ \hline
\end{tabular}
\section{Wnioski}\label{sec:end}
Podsumowując język Latex jest językiem, w którym można bardzo szybko napisać np.pracę magisterską.\newline
Nie udało mi się w sprawozdaniu umieścić importu bibliografii do pliku .bib.\newline
Jednak reszta zadań nie sprawiła mi większego problemu.

\end{document}